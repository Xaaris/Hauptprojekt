\chapter{Introduction}\label{ch:introduction}

Measuring the speed of a passing vehicle is a common practice in our society to regulate traffic and ensure road safety.
It can be achieved in a variety of ways such as with a radar gun, induction loops or cameras fixed to overarching bridges.
All those methods require large, expensive technical equipment which is not trivial to come by or to handle.
In this paper a computer vision system is proposed which takes arbitrary video files, analyzes them for vehicles and returns a velocity estimation.
It does so by measuring the height of the license plates of the passing cars, an object with known dimensions as they are standardized.

This system is a progression of a much simpler system build for the first masters project (\textit{Grundprojekt}).
The system back then was capable of locating license plates within a picture.
It did so by finding vehicles with the help of a convolutional neural network (CNN) and then applying various filters and operators on the determined region to precisely locate the license plate.

The new system described in this work improves on that in a variety of ways.
The CNN used to detect vehicles is more capable, has a higher detection rate and is still faster.
This also shows dramatically in the number of frames the system is able to process per second.
The license plate localization was completely reworked and is using a two-step approach now.
It uses Haar classifiers to propose regions where license plates might be located.
This is very fast but results in a lot of false positives.
That is why in a second step a CNN is used to validate if a region proposal really contains a plate.
This method results in a much higher hit rate compared to the first version but at the same time is a lot more efficient than a sliding window approach.
Furthermore, the exact dimensions of the detected license plate are now calculated using curve fitting to get the most accurate result.
To further reduce the margin of error camera calibration was introduced to correct each video frame for lens distortion.
