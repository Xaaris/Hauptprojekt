
\chapter{Outlook and Conclusion} \label{ch:outlook}

The results show that it is possible to estimate the speed of vehicles in videos to a fairly accurate degree using the known dimensions of the license plates.
The advantages are that only a commodity camera such as a smartphone and some computing power is needed instead of specialized equipment i.e.\ a radar or laser gun.
Disadvantages include limitations regarding low light scenarios and the measurement when multiple cars are in the video frame.
Some of those can be rather easily addressed in software as said in section~\ref{sec:limitations}, but others require additional hardware.

The overall complexity of the system makes it difficult to verify its precision and performance accurately.
Further testing is required to precisely measure the error of speed estimation, especially regarding higher speeds.

As the system is only trained on German license plates, it would need to be retrained when deploying it to a different locale, especially when license plate dimensions or colors are different.
A mechanism to measure the speed of motor-bikes would go into a similar direction.

Overall the system as it stands now is a big step up in utility, maintainability and extensibility from the first iteration built during the \textit{Grundproject}.
